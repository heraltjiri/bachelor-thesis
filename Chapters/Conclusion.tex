\chapter{Zhodnocení}
\section{Uplatněné předchozí znalosti}
Nejvíce jsem využil obsah předmětů Databázové systém I a Databázové systémy II. Tyto předměty mi na začátku pomohly se mnohem lépe zorientovat a pochopit aktuální databázový model systému. U většiny úkolů bylo potřeba aktivně pracovat s databází a často i měnit její strukturu.

Předměty Vývoj informačních systémů a Úvod do softwarového inženýrství mi pomohly s lepším pochopením a udržováním návrhových vzorů, které se ve stávajícím návrhu informačního systému nacházely. 

Jelikož je celý systém psaný objektově, neobešel bych se ani bez předmětu Objektově orientované programování.

Pro optimalizaci některých částí byly klíčové předměty Algoritmy I a II, které mi pomohly pochopit časovou i prostorovou náročnost použitých algoritmů a dále je tak optimalizovat.


\section{Scházející dovednosti}

V projektu je využita řada technologií, se kterými jsem se ve škole téměř nestkal. S některými jsem měl základní zkušenosti ještě před samotnou praxí, se zbytkem jsem se seznamoval až v průběhu. Jde třeba o JavaScript, HTML, CSS, React nebo Nette. Určitě to mělo vliv na časovou náročnost zpracování jednotlivých úkolů a pokud bych podobné zadání řešil znovu, určitě by realizace zabrala výrazně méně času.

Během výuky každopádně není možné probrat vše a osobně považuji za důležitější vývoj pochopit spíše obecně, než se učit pracovat se všemi konkrétními programovacími jazyky a knihovnami. Použití konkrétního nástroje bylo vždy možné nastudovat na ukázkových příkladech a pomocí poskytnuté dokumentace.

\chapter{Závěr}
Tato bakalářská práce si kladla za cíl především popsat celkový průběh mé individuální praxe, zadání i přístup k řešení konkrétních úkolů a také popsat kombinací technologií které se uvnitř firmy k vývoji používají.

V průběhu praxe jsem získal spoustu zkušeností s vývojem a programováním, seznámil se s novými technologiemi, naučil se lépe samostatně zpracovávat zadané úkoly a také lépe a sebevědoměji prezentovat výsledky své práce. 
Jako nejpřínosnější považuji rozšíření schopnosti si samostatně nastudovat použití různých knihoven a jejich následnou implementaci do stávajícího řešení. 

Celkově pro mě byla praxe velmi přínosná a ve spolupráci s firmou bych rád pokračoval i nádále.
\endinput
\chapter{Závěr}
V průběhu praxe jsem získal spoustu zkušeností s vývojem a programováním, seznámil se s novými technologiemi, naučil se lépe samostatně zpracovávat zadané úkoly a také lépe a sebevědoměji prezentovat výsledky své práce. 

Nejvíce jsem využil obsah předmětů Databázové systém I a Databázové systémy II. Tyto předměty mi na začátku pomohly se mnohem lépe zorientovat a pochopit aktuální databázovém model systému. U většiny úkolů bylo potřeba aktivně pracovat s databází a často i měnit její strukturu.

Předměty Vývoj informačních systémů a Úvod do softwarového inženýrství mi pomohly s lepším pochopením a udržováním návrhových vzorů, které se ve stávajícím návrhu informačního systému nacházely. 

Jelikož je celý systém psaný čistě objektově, neobešel bych se ani bez předmětu Objektově orientované programování.

V projektu je využita řada technologií, se kterými jsem se ve škole nestkal. S některými jsem měl základní zkušenosti ještě před samotnou praxí, se zbytkem jsem se seznamoval až v průběhu. Určitě to mělo vliv na časovou náročnost zpracování jednotlivých úkolů a pokud bych podobné zadání řešil znovu, určitě by realitace zabrala výrazně méně času.

Celkově pro mě byla praxe velmi přínosná a se spoluprácí s firmou bych rád pokračoval i nádále.
\endinput